\section{Ejercicio 7}
\subsection*{A)}
El scheduler fue armado a trav\'es de varias partes, en particular la isr32 que cicla las tareas. La funcion sched\_proximo\_indice (sched.c) se encarga de devolver el selector de 
la siguiente tarea para que sea ejecutada.
Los elementos necesarios para estas funciones, que deben ser inicializados, est\'an en el struct str\_game juego (game.h) y son inicializados junto a otras variables globales en 
game.c.
\begin{codesnippet}
\begin{verbatim}
typedef struct str_game_tp { 
		tarea tareasA[9];	//.p en 0 si estan libres, en 1 si estan ocpuadas (son 9 para facilitar
							//la opcion "todas ocupadas")
		tarea tareasB[9];
		unsigned int tareaActualA;	//sera la ultima (o actual) tarea del jugador
		unsigned int tareaActualB;
		unsigned int tareaRemovida;	//de 0 a 15, 0-7 = A y 8-15 = B (para el debugger)
		unsigned int filaA;			//las filas son donde se encuentra el jugador
		unsigned int filaB;
		int claseA;		//la clase que tiene seleccionada el jugador
		int claseB;
		unsigned int jugadorActual; //para saber que jugador esta en el tic de reloj actual
		int zombis_restantes_a; // 8 menos esto es la cantidad que se pueden lanzar.
		int zombis_restantes_b;
		int puntaje_A;
		int puntaje_B;
		unsigned int debugger;		//flag debugger
		unsigned int _20A;			//zombies que le quedan por lanzar
		unsigned int _20B;
} __attribute__((__packed__)) str_game;
\end{verbatim}
\end{codesnippet}
\begin{codesnippet}
\begin{verbatim}
void inicializarJuego()
{
	estaIdle = 1;
	tiempo_esperando = 0;
	juego.zombis_restantes_a = 8;
	juego.zombis_restantes_b = 8;
	juego._20A = 20;
	juego._20B = 20;
	juego.puntaje_A = 0;
	juego.puntaje_B = 0;
	
	puntaje_o_restantes(juego._20A, 31, 47, 0x4f);
	puntaje_o_restantes(juego._20B, 49, 47, 0x1f);

	puntaje_o_restantes(juego.puntaje_A , 36, 47, 0x4f);
	puntaje_o_restantes(juego.puntaje_B , 41, 47, 0x1f);

	int i = 0;
	for (i = 0; i < 9; i++)
	{
		juego.tareasA[i].p = 0;
		juego.tareasB[i].p = 0;
	}	
	juego.tareaActualA = 0;
	juego.tareaActualB = 0;
	juego.tareaRemovida = 16; //0-7 = A, 8-15 = B, 16 invalido (no se removio ninguna tarea)
	juego.filaA = 1;
	juego.filaB = 1;
	juego.claseA = 0;
	juego.claseB = 0;
	juego.jugadorActual = 0;
	juego.debugger = 0;
}
\end{verbatim}
\end{codesnippet}

\subsection*{B)}
sched\_proximo\_indice devolver\'a el selector de la siguiente tarea a ejecutar. Tiene en cuenta las distintas combinaciones de cantidades de zombies 
y trabaja con las variables globales de \textbf{juego}.
Lo que hace sched\_proximo\_indice es revisar a quien le toca jugar (en la variable juego.jugadorActual se encuentra el \'ultimo en jugar, asi que 
debe tomar el contrario), y de \'ese jugador revisar\'a cual es el zombie para lanzar. En caso de que no haya ninguno, lanzar\'a un zombie del jugador 
contrario, y en caso de que ninguno tenga zombies devolver\'a $0$ para que no salte a ninguna tarea (porque siempre la anterior tarea ser\'a la Idle).
Si el jugador del turno actual no tiene zombies y se lanza un zombie del jugador contrario, se avisar\'a por medio de la variable juego.jugadorActual 
este hecho para que no se salteen turnos en ning\'un caso.
\begin{codesnippet}
\begin{verbatim}
if (juego.jugadorActual == 0) 
	{	//si el actual es A, ahora pasa a jugar B 
		res = sched_indice_B();
		if (res == 8) {	//si no hay zombies de B...
			res = sched_indice_A();
			if (res == 8) { //si no hay zombies de A..
				res = 0;
			} else {	//si hay zombies de A..
				if (juego.tareaActualA == res && !estaIdle) {
						//si hay un solo zombie de A..
					res = 0;
				} else {	
					juego.tareaActualA = res;
					res = (res + GDT_IDX_TSS_A0) *8;
				}
				
				juego.jugadorActual = 1;
			}
		} else {	//si hay zombies de B..
			juego.tareaActualB = res;
			res = (res + GDT_IDX_TSS_B0) *8;
		}
	}
\end{verbatim}
\end{codesnippet}

\subsection*{C)}
Para este punto se modific\'o el funcionamiento de la rutina de atencion de int 0x66 (isr102) que se tenia en el punto 5. Ahora, se va a encargar de 
mover al zombie de la tarea 
que se esta ejecutando. La interrupci\'on llama a game\_move\_current\_zombi. Una vez realizada la acci\'on correspondiente se procede a saltar a la 
tarea idle, si no se esta ejecutando por alguna raz\'on (como puede ser que tire una excepci\'on).

isr102 recibir\'a por $eax$ la direcci\'on a la que se debe mover. game\_move\_current\_zombi lo toma y calcula la nueva posici\'on. Guarda la 
vieja posici\'on para saber de donde copiar el c\'odigo de la tarea zombie.

game\_move\_current\_zombi toma en cuenta los casos de que el zombie llegue a la fila superior o inferior, ya que el mapa es cilindrico. Tambi\'en 
ve los casos en los que llega a uno u otro extremo, ya que debe desalojar la tarea (con .p = 0) y darle el punto al jugador correspondiente.

\begin{codesnippet}
\begin{verbatim}
_isr102:
	cli
	pushad
	push eax
	call game_move_current_zombi
	cmp BYTE [estaIdle], 0x1
	je .fin
	mov dword [estaIdle], 0x1
	mov dword [selector], 0x70
	
	jmp far [offset]
	jmp .fin
.fin:
	pop eax
	popad
	iret
\end{verbatim}
\end{codesnippet}

\subsection*{D)}
La atenci\'on de la interrupci\'on del reloj saltar\'a siempre que haya al menos un zombie (si no hay, sched_proximo_indice devolver\'a 0). 
cambiar\_jug\_actual modifica 
juego.jugadorActual. El cambio de tareas, que se realiza dentro de la isr32 tiene los casos de cambio de tarea a la idle, a un zombie o no 
saltar (siempre que se deba saltar a la tarea que se est\'a corriendo en ese momento).


Ademas, si en un tiempo determinado (1000 ticks) ning\'un jugador realiza un lanzamiento (a\'un habiendo zombies por lanzar) y 
ning\'un zombie se mueve, el juego finaliza utilizand\'o como criterio de ganador el valor que se tenga hasta el momento de $puntaje\_A$ y 
$puntaje\_B$. Para esto, cada vez que se produzca un tick de reloj (fuera del modo debugger) se 
increment\'a una variable global llamada $tiempo\_esperando$ (inicializada en cero). Pero, si un jugador lanza un zombie o se produce un movimiento de 
los mismos vuelve al 
valor cero. En caso de que se llegue al valor 1000 se llama a la funci\'on $ganador$, la misma verifica el valor de par\'ametro de entrada. Si es 1, 
que es este caso, se 
muestra por pantalla quien es el ganador y se setea el valor de $\_20A$ y $ \_20B$ como cero, para indicar que ya no se pueden lanzar m\'as zombies y 
dar por finalizado el 
juego. Si es otro valor y $\_20A$ y $ \_20B$ son iguales a cero se lanzaron todos los zombies y se muestra al ganador. En cualquiera de los dos 
casos, luego de finalizar el juego se salta a la tarea idle.

Este m\'etodo de parada fue diseñado por el hecho de que una tarea puede ser modificada por otra tarea en el medio de campo, y como consecuencia 
dejar de moverse.

		
\begin{codesnippet}
\begin{verbatim}
_isr32:
	cli
	pushad
		;prioridad: si estamos en el debugger no se toca nada
		mov eax, 1
		cmp eax, [Debugger]
		je .debugger
			;si no estamos en el debugger, y no se toco nada hace 1000 
			;ciclos de reloj, asumo que el juego termino.
		inc dword[tiempo_esperando]
		cmp dword [tiempo_esperando], 1000
		jge .termino
	;dibujo los clocks
	call clock
	call proximo_reloj
	;paso a la tarea siguiente (o me quedo en esta si son la misma
	call sched_proximo_indice
	push eax
	call cambiar_jug_actual
	pop eax
	cmp ax, 0
	je .noJump
	mov [selector], ax
	mov dword [estaIdle], 0
	call fin_intr_pic1
	jmp far [offset]
	jmp .end
.termino:
	xor eax, eax
	inc eax
	push eax
	call ganador
	pop eax
.debugger:
	cmp byte [estaIdle], 1
	je .noJump
	mov dword [estaIdle], 0x1
	mov dword [selector], 0x70
	jmp far [offset]
.noJump:
	call fin_intr_pic1
.end:
	popad
	iret
	iret
\end{verbatim}
\end{codesnippet}

\subsection*{E)}
El desalojo de las tareas al producirse una excepci\'on se realiza en $desalojar\_tarea$, la cual desaloja simplmente poniendo 0 en la propiedad 
$p$ de la tarea, en el struct $juego$. Adem\'as de anularla habilita al jugador de la tarea desalojada a lanzar un nuevo zombie.

\begin{codesnippet}
\begin{verbatim}
%macro ISR 1
global _isr%1

_isr%1:
    mov EAX, %1
    PUSH EAX
    CALL mostrarError
    pop eax
.fin:
    call desalojar_tarea
    
    mov dword [estaIdle], 0x1
    mov dword [selector], 0x70
	jmp far [offset]
    iret
%endmacro
\end{verbatim}
\end{codesnippet}

\subsection*{F)} El mecanismo de debugging utilizar\'a en primer lugar juego.tareaRemovida, variable que devolver\'a qu\'e tarea es de la que tiene que mostrar informaci\'on, 
y juego.debugger que servir\'a de flag para indicar si debe activarse el debugger o no. Esta variable se modifica al pulsar la tecla 'y'.
La funci\'on principal que se encarga de controlar el proceso es mDebugger(), la cual se encarga de copiar el sector de la pantalla a una variable global para tal fin, y luego 
cargar el fondo y las etiquetas de cada registro.
Los datos son cargados con cargoDebugger(), que lee los datos de la tss de la tarea reci\'en desalojada. Luego, se queda en la tarea Idle hasta que una 
interrupci\'on de teclado hace salir del bucle. 
\begin{codesnippet}
\begin{verbatim}
void mDebugger()
{
	//guarda la pantalla, pega la pantalla y luego llama a una funcion en asm q imprime el estado de los registros
	//guardo columnas 25-55, filas 6-42 (36 filas, 30 cols)
	if (juego.debugger)
	{
		guardoPantalla();
		Debugger = 1;

	//[...] dibujo la pantalla del debugger
		
		cargoDebugger();
	}
}
\end{verbatim}
\end{codesnippet}

\begin{codesnippet}
\begin{verbatim}
void cargoDebugger()
{
	unsigned int base2;
	base2 = (unsigned int)gdt[juego.tareaRemovida + GDT_IDX_TSS_A0].base_0_15;
	base2 += (unsigned int)gdt[juego.tareaRemovida + GDT_IDX_TSS_A0].base_23_16 << 16;
	base2 += (unsigned int)gdt[juego.tareaRemovida + GDT_IDX_TSS_A0].base_31_24 << 24;

	tss* base = (tss*)base2;
	//base apunta a la tss que desaloje
	if (juego.tareaRemovida < 8)
	{
		switch (juego.tareasA[juego.tareaActualA].clase)
		{
			case 0:
				print("Zombie  A  Guerrero", 26, 7, 0x1f);
				break;			
			case 1:
				print("Zombie  A  Mago", 26, 7, 0x1f);
				break;			
			case 2:
				print("Zombie  A  Clerigo", 26, 7, 0x1f);
				break;			
		}
	}
	else
	{
		switch (juego.tareasB[juego.tareaActualB].clase)
		{
			case 0:
				print("Zombie  B  Guerrero", 26, 7, 0x1f);
				break;			
			case 1:
				print("Zombie  B  Mago", 26, 7, 0x1f);
				break;			
			case 2:
				print("Zombie  B  Clerigo", 26, 7, 0x1f);
				break;			
		}
	}
	
//	[...] escribo en pantalla el valor de los registros pedidos y la pila de la tarea
}
\end{verbatim}
\end{codesnippet}

Es importante notar que luego de cargarse el debugger, al estar la variable Debugger en 1, la interrupci\'on isr32 (reloj) no cambiar\'a de tareas y 
se quedar\'a en la idle (o cambiar\'a a esta de encontrarse en otra tarea) hasta que se produzca una interrupci\'on de teclado. En la funci\'on 
\textbf{teclado}, si la interrupci\'on fue dada por precionar una tecla, actualizar\'a el valor de la variable Debugger en 0, haciendo que se 
cargue el sector de la pantalla que fue removido al pasar al modo debugger.


