\section{Ejercicio 2}
\subsection*{A)}
Utilizamos la macro dada en isr.asm, que dado un entero crea un c\'odigo \_isr(). Esta funci\'on llamar\'a a la funcion en C $mostrarError$, luego desalojar\'a la  tarea actual y 
por ultimo pasar\'a a la tarea idle, si esta no es la tarea actual.

\begin{codesnippet}
\begin{verbatim}
%macro ISR 1
global _isr%1

_isr%1:
    mov EAX, %1
    PUSH EAX
    CALL mostrarError
    pop eax
.fin:
    call desalojar_tarea
    
    mov dword [estaIdle], 0x1
    mov dword [selector], 0x70
	jmp far [offset]
    iret
%endmacro
\end{verbatim}
\end{codesnippet}

MostrarError es un switch case, que dado un entero entre 0 y 20 mapea un error a un mensaje(descripci\'on) y lo imprime por pantalla con la funci\'on print dada en screen.c, 
y por ultimo llama a la funcion $mDebugger()$.
Luego utilizamos la macro ISR para $i$, $i= 0..20$
En isr.h llamamos a \_isr($i$), para $i= 0..20$, ya que esto ser\'a utilizado por idt.c, para obtener el offset de la rutina encargada de atender la interrupci\'on. \newline
Por \'ultimo en idt.c inicializamos el arreglo de interrupciones con la funci\'on $idt\_inicializar$. Para esto, a su vez inicializamos a cada interrupci\'on dentro del mismo 
por medio de una macro, \#$define$ $IDT\_ENTRY(numero)$. La misma se encarga de completar todos los campos que poseen los descriptores de segmento de las interrupciones. 
Un offset que es la direcci\'on de la funci\'on \_isr correspondiente; segsel que se corresponde al segmento de c\'odigo nivel cero. Y los atributos 
P=1; dpl correspondiente a la tarea y d= 1 (para 32bits). 

\subsection*{B)}
En kernel.asm llamamos a idt\_inicializar(función en $idt.c$), que completa el vector de interupciones de las posiciones 0 a 19, 32, 33 y 102 con \#define IDT\_ENTRY(numero) de la 
forma descripta en el punto $A$.

 
 
  La entrada 102 es la interrupción $mover$, la cual es llamada por las tareas. Por eso pusismos los atributos correspondientes (0xEE00) para que sea de 
nivel 3\color{black} . Luego con LIDT cargamos el vector de interrupciones(idt_entry en $idt.h$) utilizando la estructura IDT_DESC definido en $idt.c$.
